\documentclass[prb,preprint]{revtex4-2}

\usepackage{amsmath}    % need for subequations
\usepackage{graphicx}   % need for figures
\usepackage{verbatim}   % useful for program listings
\usepackage{color}      % use if color is used in text
\usepackage{subfigure}  % use for side-by-side figures
\usepackage{hyperref}   % use for hypertext links, including those to external
\raggedbottom           % don't add extra vertical space
\begin{comment}
\end{comment}

\bibliographystyle{apsrev4-2}

\begin{document}

\title{
    Machine-learned embedded atom method
}

\author{Xin Chen, 
        Li-Fang Wang, 
        De-Ye Lin$^\mathrm{*}$, 
        Hai-Feng Song$^\mathrm{*}$}

\affiliation{
    Institute of Applied Physics and Computational Math, 
    Beijing 100088, China}

\affiliation{
    CAEP Software Center for High Performance Numerical Simulation, 
    Beijing 100088, China}

\maketitle

% % % % % % % % % % % % % % % % % % % % % % % % % % % % % % % % % % % % % % % %
% 
% Appendix
%
% % % % % % % % % % % % % % % % % % % % % % % % % % % % % % % % % % % % % % % %
\section*{Appendix}
\label{sec:appendix}

\subsection{The stress equation}

\newcommand{\dE}{\partial E^{total}}
\newcommand{\hab}{h_{\alpha\beta}}
\newcommand{\hga}{h_{\gamma\alpha}}
\newcommand{\hgb}{h_{\gamma\beta}}
\newcommand{\hgd}{h_{\gamma\delta}}
\newcommand{\rijn}{r_{ij\mathbf{n}}}

Assume we have:
\begin{align}
\mathbf{h} = & \begin{pmatrix}
    h_{xx} & h_{xy} & h_{xz} \\
    h_{yx} & h_{yy} & h_{yz} \\
    h_{zx} & h_{zy} & h_{zz} \\
\end{pmatrix} \\
\mathbf{n} = & \begin{pmatrix}
    n_{x} \\
    n_{y} \\
    n_{z}
\end{pmatrix}
\end{align}
where $\mathbf{h}$ is the $3 \times 3$ cell tensor and $\mathbf{n}$ represents
the periodic boundary shift vector. Then we can get:
\begin{align}
\frac{\partial \rijn}{\partial \hab} = &
    \frac{1}{\rijn} \cdot \Delta_{ij\mathbf{n}\beta} \cdot n_{\alpha} \\
\Delta_{ij\mathbf{n}\beta} = & r_{j,\beta}^{(0)} - r_{i,\beta}^{(0)} + 
    \sum_{\alpha}{n_{\alpha}h_{\alpha \beta}}
\end{align}
So we can compute the derivative of $E^{total}$ with respect to $\hab$:
\begin{align}
\frac{\dE}{\partial \hab} = & \sum_{i}^N{
    \frac{\partial E_{i}}{\partial \hab}
} = 
\sum_{i}^{N}{\frac{\partial \left(
    \sum_{j}{\sum_{\mathbf{n}}
        \phi_{ij}(\rijn)
    } + F_{i}\left( \sum_{j}{\sum_{\mathbf{n}}{
        \rho_{ij}(\rijn)
    }} \right)
\right)}{\partial \hab}} \nonumber \\
= & \sum_{i}^{N}{\left(
    \sum_{j}{\sum_{\mathbf{n}}{
        \frac{\partial \phi_{ij}(\rijn)}{\partial \hab}
    }} + \frac{\partial F_{i}(\rho_{i})}{\partial \rho_{i}} \cdot 
    \sum_{j}{\sum_{\mathbf{n}}{\frac{\partial \rho_{ij}(\rijn)}{\partial \hab}}}
\right)} \nonumber \\
= & \sum_{i}^{N}{
    \sum_{j}{\sum_{\mathbf{n}}{\left(
        \frac{\partial \phi_{ij}(\rijn)}{\partial \hab} +
        \frac{\partial F_{i}(\rho_{i})}{\partial \rho_{i}} \cdot 
        \frac{\partial \rho_{ij}(\rijn)}{\partial \hab}
    \right)}}
} \nonumber \\
= & \sum_{i}^{N}{
    \sum_{j}{\sum_{\mathbf{n}}{\left(
        \frac{\partial \phi_{ij}(\rijn)}{\partial \rijn} +
        \frac{\partial F_{i}(\rho_{i})}{\partial \rho_{i}} \cdot 
        \frac{\partial \rho_{ij}(\rijn)}{\partial \rijn}
    \right) \cdot 
    \frac{\partial \rijn}{\partial \hab}
}}} \nonumber \\
= & \sum_{i}^{N}{
    \sum_{j}{\sum_{\mathbf{n}}{\left(
        \frac{\partial \phi_{ij}(\rijn)}{\partial \rijn} +
        \frac{\partial F_{i}(\rho_{i})}{\partial \rho_{i}} \cdot 
        \frac{\partial \rho_{ij}(\rijn)}{\partial \rijn}
    \right) \cdot 
    \frac{1}{\rijn} \cdot \Delta_{ij\mathbf{n}\beta} \cdot n_{\alpha}
}}} \nonumber \\
= & -\sum_{i}^{N}{
    \sum_{j}{\sum_{\mathbf{n}}{f_{ij\mathbf{n}\beta} \cdot n_{\alpha}
}}}
\end{align}
where $f_{ij\mathrm{n}\beta}$ is the partial force:
\begin{equation}
f_{ij\mathrm{n}\beta} = \left(
    \frac{\partial \phi_{ij}(\rijn)}{\partial \rijn} +
    \frac{\partial F_{i}(\rho_{i})}{\partial \rho_{i}} \cdot 
    \frac{\partial \rho_{ij}(\rijn)}{\partial \rijn}
\right) \cdot 
\frac{1}{\rijn} \cdot \Delta_{ij\mathbf{n}\beta}
\end{equation}
Then, we can have:
\begin{align}
\label{eq:dEdhTh_g2_expanded}
\left(
    \left(\frac{\dE}{\partial \mathbf{h}}\right)^T \mathbf{h}
\right)_{\alpha\beta} = &
\sum_{\gamma}{\frac{\dE}{\partial \hga} \hgb} \nonumber \\
= & -\sum_{\gamma}{\sum_{i}^{N}{
        \sum_{j}{\sum_{\mathbf{n}}{f_{ij\mathbf{n}\alpha} \cdot n_{\gamma}
    }}} \cdot \hgb
} \nonumber \\
= & -\sum_{i}^{N}{
    \sum_{j}{\sum_{\mathbf{n}}{f_{ij\mathbf{n}\alpha} \cdot \sum_{\gamma}{ 
        n_{\gamma} \hgb
    }
}}}
\end{align}

Thus, the virial stress tensor $\epsilon$ can be expressed with a simpler form:
\begin{equation}
\epsilon = -F^{T}R + \left(\frac{\dE}{\partial \mathbf{h}}\right)^T \mathbf{h}
\end{equation}
where $F$ is the $N \times 3$ total forces matrix and $R$ is the $N \times 3$ 
positions matrix.

\subsection{The elastic constant tensor}

% First, we compute the second-order derivative of $E^{total}$ with respect to 
% $\mathbf{h}$:
% \begin{align}
% \frac{\partial^2 E}{\partial \hab \partial \hgd}
% = & \sum_{i}^{N}{
%     \sum_{j}{\sum_{\mathbf{n}}{\frac{\partial\left(
%         \phi_{ij}^{'}(\rijn) + F_{i}^{'}(\rho_{i}) \rho_{ij}^{'}(\rijn)
%     \right) \cdot 
%     \frac{\Delta_{ij\mathbf{n}\beta} \cdot n_{\alpha}}{\rijn} 
% }{\partial \hgd}}}}
% \end{align}
% where
% \begin{align}
% \frac{\partial \phi_{ij}^{'}(\rijn)}{\partial \hgd} \cdot 
% \frac{\Delta_{ij\mathbf{n}\beta} \cdot n_{\alpha}}{\rijn} 
% = & \phi_{ij}^{''}(\rijn) \cdot 
% \frac{\Delta_{ij\mathbf{n}\beta}\Delta_{ij\mathbf{n}\delta} \cdot 
% n_{\alpha}n_{\gamma}}{\rijn^2} \\
% \frac{\partial F_{i}^{'}(\rho_{i}) \rho_{ij}^{'}(\rijn)}{\partial \hgd} \cdot 
% \frac{\Delta_{ij\mathbf{n}\beta} \cdot n_{\alpha}}{\rijn} 
% = & F_{i}^{'}(\rho_{i}) \rho_{ij}^{''}(\rijn) \cdot 
% \frac{\Delta_{ij\mathbf{n}\beta}\Delta_{ij\mathbf{n}\delta} \cdot 
% n_{\alpha}n_{\gamma}}{\rijn^2} \\
% \frac{\partial \Delta_{ij\mathbf{n}\beta}}{\partial \hgd}
% = & n_{\beta} \\
% \frac{\partial \Delta_{ij\mathbf{n}\delta}}{\partial \hgd}
% = & n_{\delta} \\
% \frac{\partial \rijn^{-1}}{\partial \hgd}
% = & -\frac{1}{\rijn^2} \cdot \frac{\Delta_{ij\mathbf{n}\delta} \cdot 
% n_{\gamma}}{\rijn}
% \end{align}
% Thus, we can have:
% \begin{align}
% \frac{\partial^2 E}{\partial \hab \partial \hgd} = &
% \end{align}
The elastic constant tensor $C^{\alpha\beta\gamma\delta}$ can be computed with 
the following equation:
\begin{equation}
C^{\alpha\beta\gamma\delta} = \frac{1}{V} \cdot \left(
    \frac{\partial\epsilon}{\partial\mathbf{h}}
\right)^ {\alpha\beta\gamma\eta} \mathbf{h}^{\eta\delta}
\end{equation}
where $\alpha, \beta, \gamma, \eta, \delta = x, y, z$, $\epsilon$ is the stress
tensor, $\mathbf{h}$ is the lattice tensor and $V$ is the volume.

\end{document}